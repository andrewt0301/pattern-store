%%%%%%%%%%%%%%%%%%%%%%%%%%%%%%%%%%%%%%%%%%%%%%%%%%%%%%%%%%%%%%%%%%%%%%%%%%%%%%%%
%2345678901234567890123456789012345678901234567890123456789012345678901234567890
%        1         2         3         4         5         6         7         8

\documentclass[letterpaper, 10 pt, conference]{ieeeconf}  % Comment this line out
                                                          % if you need a4paper
%\documentclass[a4paper, 10pt, conference]{ieeeconf}      % Use this line for a4
                                                          % paper

\IEEEoverridecommandlockouts                              % This command is only
                                                          % needed if you want to
                                                          % use the \thanks command
\overrideIEEEmargins
% See the \addtolength command later in the file to balance the column lengths
% on the last page of the document



% The following packages can be found on http:\\www.ctan.org
%\usepackage{graphics} % for pdf, bitmapped graphics files
%\usepackage{epsfig} % for postscript graphics files
%\usepackage{mathptmx} % assumes new font selection scheme installed
%\usepackage{times} % assumes new font selection scheme installed
%\usepackage{amsmath} % assumes amsmath package installed
%\usepackage{amssymb}  % assumes amsmath package installed

\title{\LARGE \bf
Client-Server Application for Storage Administering of Source Code Change Patterns
}

\author{
Alexander Krasnov \\ 
\textit{Faculty of Computer Science, Higher School of Economics} \\
Moscow, Russia \\
aakrasnov@edu.hse.ru
}


\begin{document}



\maketitle
\thispagestyle{empty}
\pagestyle{empty}


%%%%%%%%%%%%%%%%%%%%%%%%%%%%%%%%%%%%%%%%%%%%%%%%%%%%%%%%%%%%%%%%%%%%%%%%%%%%%%%%
\begin{abstract}

The majority of contemporary projects use third-party libraries which update from one version to another one takes a lot of time and requires repetition of monotonous actions. This process could be simplified by using already existing patterns that are combined into documents. How such documents are distributed? There is no central storage for documents with kind of information, therefore it is necessary to develop a storage to ensure convenient interaction with available data. This paper describes the client-server application for storage administering of source code change patterns.

\end{abstract}

\begin{keywords} 
storage, patterns, library migration, administering
\end{keywords}


%%%%%%%%%%%%%%%%%%%%%%%%%%%%%%%%%%%%%%%%%%%%%%%%%%%%%%%%%%%%%%%%%%%%%%%%%%%%%%%%
\section{Introduction}

TODO: INTRO HERE

\section{Literature review}
Literature reference: \cite{c1}
\subsection{Possible subsection name}

TODO: subsection example if necessary

\section{Methodology}

TODO: methodology

\begin{itemize}
\item first item
\item second item
\end{itemize}

\section{Expected results}

TODO: expected results

\section{Conclusion}

TODO: Conclusion

\addtolength{\textheight}{-12cm}   % This command serves to balance the column lengths
                                  % on the last page of the document manually. It shortens
                                  % the textheight of the last page by a suitable amount.
                                  % This command does not take effect until the next page
                                  % so it should come on the page before the last. Make
                                  % sure that you do not shorten the textheight too much.

%%%%%%%%%%%%%%%%%%%%%%%%%%%%%%%%%%%%%%%%%%%%%%%%%%%%%%%%%%%%%%%%%%%%%%%%%%%%%%%%



%%%%%%%%%%%%%%%%%%%%%%%%%%%%%%%%%%%%%%%%%%%%%%%%%%%%%%%%%%%%%%%%%%%%%%%%%%%%%%%%



%%%%%%%%%%%%%%%%%%%%%%%%%%%%%%%%%%%%%%%%%%%%%%%%%%%%%%%%%%%%%%%%%%%%%%%%%%%%%%%%



\begin{thebibliography}{99}
\bibitem{delete} TODO: FIX REFERENCES
\bibitem{c1} International Journal of Computer Applications (0975 – 8887)
Volume 31– No.8, October 2011
\bibitem{c2} Planning for Deep Space Network Operations
Randall Hill, Jr., Steve Chien, Crista Smyth, Kristina Fayyad, Trish Santos
Jet Propulsion Laboratory
California Institute of Technology.
\bibitem{c3} Automating Mid- and Long-Range Scheduling for the NASA Deep Space Network
Mark D. Johnston* and Daniel Tran*
*Jet Propulsion Laboratory, California Institute of Technology
4800 Oak Grove Drive, Pasadena CA USA 91109

\bibitem{c4} A Survey of Transport Protocols for Deep Space
Communication Networks

International Journal of Computer Applications (0975 – 8887)
Volume 31– No.8, October 2011
\bibitem{c5} H. Balakrishnan, V. N. Padmanabhan and R. H. Katz, “The Effects of Asymmetry on TCP Performance," Proc. ACM MOBICOM Hungary, pp. 77-89, September 1997.
\bibitem{c6} M.Allman, D.Glover, and L.Sanchez, “Enhancing TCP over satellite channels using standard mechanisms,” IETF, RFC 2488, January 1999.
\bibitem{c7}S.Burleigh, A.Hooke, et al., “Delay-Tolerant Networking: An Approach to Interplanetary Internet,” IEEE Communications Magazine, Vol.41, Issue 6, pp. 128-136, June 2003.
\bibitem{c8} O. B. Akan, J.Fang and I. F. Akyildiz, “TP-Planet: A Reliable Transport Protocol for InterPlaNetary Internet”, IEEE/SAC, Vol. 22, No. 2, pp 348-61, February 2004.


 \bibitem{c9} M. Young, The Techincal Writers Handbook. Mill Valley, CA: University
Science, 1989.



\end{thebibliography}

\end{document}

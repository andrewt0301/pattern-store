%%%%%%%%%%%%%%%%%%%%%%%%%%%%%%%%%%%%%%%%%%%%%%%%%%%%%%%%%%%%%%%%%%%%%%%%%%%%%%%%
%2345678901234567890123456789012345678901234567890123456789012345678901234567890
%        1         2         3         4         5         6         7         8

\documentclass[letterpaper, 10 pt, conference]{ieeeconf}  % Comment this line out
                                                          % if you need a4paper
%\documentclass[a4paper, 10pt, conference]{ieeeconf}      % Use this line for a4
                                                          % paper

\IEEEoverridecommandlockouts                              % This command is only
                                                          % needed if you want to
                                                          % use the \thanks command
\overrideIEEEmargins
% See the \addtolength command later in the file to balance the column lengths
% on the last page of the document



% The following packages can be found on http:\\www.ctan.org
%\usepackage{graphics} % for pdf, bitmapped graphics files
%\usepackage{epsfig} % for postscript graphics files
%\usepackage{mathptmx} % assumes new font selection scheme installed
%\usepackage{times} % assumes new font selection scheme installed
%\usepackage{amsmath} % assumes amsmath package installed
%\usepackage{amssymb}  % assumes amsmath package installed
\usepackage{indentfirst}

\title{\LARGE \bf
Client-Server Application for Storage Administering of Source Code Change Patterns
}

\author{
Alexander Krasnov \\ 
\textit{Faculty of Computer Science, Higher School of Economics} \\
Moscow, Russia \\
aakrasnov@edu.hse.ru
}


\begin{document}



\maketitle
\thispagestyle{empty}
\pagestyle{empty}


%%%%%%%%%%%%%%%%%%%%%%%%%%%%%%%%%%%%%%%%%%%%%%%%%%%%%%%%%%%%%%%%%%%%%%%%%%%%%%%%
\begin{abstract}

The majority of contemporary projects use third-party libraries which update from one version to another one takes a lot of time and requires repetition of monotonous actions. This process could be simplified by using already existing patterns that are combined into documents. How such documents are distributed? There is no central storage for documents with kind of information, therefore it is necessary to develop a storage to ensure convenient interaction with available data. This paper describes the client-server application for storage administering of source code change patterns.

\end{abstract}

\begin{keywords} 
storage, patterns, library migration, administering
\end{keywords}


%%%%%%%%%%%%%%%%%%%%%%%%%%%%%%%%%%%%%%%%%%%%%%%%%%%%%%%%%%%%%%%%%%%%%%%%%%%%%%%%
\section{Introduction}
During the project lifecycle, developers often face the following
typical tasks: replacing used libraries, correcting common errors,
and restructuring the code according to certain rules. Completing
these tasks is time-consuming and error-prone, although it is
possible to automate this process by applying patterns based on an
abstract syntax tree \cite{c1}. Automation of routine actions using templates
will increase the productivity of developers and the quality of the project
source code.

The obvious questions that arise in this case are where to store and
how to distribute these changes. For this purpose a central storage
will be developed. It is expected that templates can come from
different sources and be specific to a project (for example, a pet
project) or a group of projects. Moreover, the set of patterns will
be constantly updated. These assumptions contribute to provide an opportunity
for storing templates, updating them and leaving feedback about usage of
existing ones.

The remainder of the paper is organized in a following way. In
Section 2 related literature is reviewed. In Section 3, the
methodology is described. Section 4 contains discussion about the expected
results. The last Section 5 summarizes the paper.
\\ TODO: INTRO HERE

\section{Literature review}
\\ TODO: REVIEW HERE
\subsection{Possible subsection name}

TODO: subsection example if necessary

\section{Methodology}

The main idea is to work out a centralized storage which will contain
source code change patterns based on an abstract syntax tree. In
general they are useful only in the case of using a certain set of
them. For this purpose patterns are combined into documents which
contain some metadata information. It includes a unique identifier,
the name of the programming language for which this document is used,
as well as the application scenario. This scenario contains type
information (migration, refactoring or an unknown type), whereas each
pattern also has several descriptive fields, which allow to identify
it. Moreover, it is possible to get information about the creation
time, the author of the pattern and some specific metadata.

The whole workflow looks in a following way. Developers create
different examples of patterns using such approaches as ...(ref1),
...(ref2), ...(ref3). After that users must be authenticated in order
to upload mined patterns and have comply with the set limits to be
able to perform this operation. They should specify metadata
information to create a new document with these patterns. 

As for an administrator of the system, he can limit the number of
uploaded patterns, specify some restrictions for the maximum amount of
uploaded data or just prohibit the upload to a specific user. Another
available functionality for an administrator is an opportunity to
modify existing documents by merging some of them, updating and
deleting.

Asking for upload the same documents is quite common for the process
of applying patterns from them, therefore it was decided to develop a
local client cache. Further details of the operation of the cache will
be given here:
\begin{itemize}
    \item It would be great to get a document from the
local cache if there is no connection to the service and the necessary
item exists.
    \item If the service is available and the item exists in the
local memory, it is necessary to check that the data has not expired
(for example, the local timestamp of the document and the timestamp of
the deleted document match).
    \item If cached documents have been modified locally in some way
and these changes have been transferred to the service, it is
necessary to delete such documents from the local cache.
    \item There is no need to validate the local cache very often. A
document can be considered fresh (e.g. not expired) if it was
updated within the specified time interval (for example, 1 hour).
Immediate refreshing of documents is not critical. It is better to
avoid redundant requests to the server.
\end{itemize}

The most common way of using the service is to download existing
documents from the storage. A user has several filters to obtain only
necessary documents with patterns. For example, he can specify the
full description of the library to be replaced during migration, and
the same for the second one. Another available filter is to get all
the data for a specific programming language. After downloading
documents from a centralized repository, patterns are applied using
the Patternika tool \cite{c1}. 

\section{Expected results}

TODO: expected results

\section{Conclusion}

TODO: Conclusion

\addtolength{\textheight}{-12cm}   % This command serves to balance the column lengths
                                  % on the last page of the document manually. It shortens
                                  % the textheight of the last page by a suitable amount.
                                  % This command does not take effect until the next page
                                  % so it should come on the page before the last. Make
                                  % sure that you do not shorten the textheight too much.

%%%%%%%%%%%%%%%%%%%%%%%%%%%%%%%%%%%%%%%%%%%%%%%%%%%%%%%%%%%%%%%%%%%%%%%%%%%%%%%%



%%%%%%%%%%%%%%%%%%%%%%%%%%%%%%%%%%%%%%%%%%%%%%%%%%%%%%%%%%%%%%%%%%%%%%%%%%%%%%%%



%%%%%%%%%%%%%%%%%%%%%%%%%%%%%%%%%%%%%%%%%%%%%%%%%%%%%%%%%%%%%%%%%%%%%%%%%%%%%%%%



\begin{thebibliography}{99}
\bibitem{delete} TODO: FIX REFERENCES
\bibitem{c1} Andrei Tatarnikov et al., “Patternika: A
Pattern-Mining-Based Tool for Automatic Library Migration”,
Proceedings of the 32nd International Symposium on Software
Reliability Engineering (ISSRE 2021), Wuhan, China, 2021, p. 6.
\bibitem{c2} Planning for Deep Space Network Operations
Randall Hill, Jr., Steve Chien, Crista Smyth, Kristina Fayyad, Trish Santos
Jet Propulsion Laboratory
California Institute of Technology.
\bibitem{c3} Automating Mid- and Long-Range Scheduling for the NASA Deep Space Network
Mark D. Johnston* and Daniel Tran*
*Jet Propulsion Laboratory, California Institute of Technology
4800 Oak Grove Drive, Pasadena CA USA 91109

\bibitem{c4} A Survey of Transport Protocols for Deep Space
Communication Networks

International Journal of Computer Applications (0975 – 8887)
Volume 31– No.8, October 2011
\bibitem{c5} H. Balakrishnan, V. N. Padmanabhan and R. H. Katz, “The Effects of Asymmetry on TCP Performance," Proc. ACM MOBICOM Hungary, pp. 77-89, September 1997.
\bibitem{c6} M.Allman, D.Glover, and L.Sanchez, “Enhancing TCP over satellite channels using standard mechanisms,” IETF, RFC 2488, January 1999.
\bibitem{c7}S.Burleigh, A.Hooke, et al., “Delay-Tolerant Networking: An Approach to Interplanetary Internet,” IEEE Communications Magazine, Vol.41, Issue 6, pp. 128-136, June 2003.
\bibitem{c8} O. B. Akan, J.Fang and I. F. Akyildiz, “TP-Planet: A Reliable Transport Protocol for InterPlaNetary Internet”, IEEE/SAC, Vol. 22, No. 2, pp 348-61, February 2004.


 \bibitem{c9} M. Young, The Techincal Writers Handbook. Mill Valley, CA: University
Science, 1989.



\end{thebibliography}

\end{document}
